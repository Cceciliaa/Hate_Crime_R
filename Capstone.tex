\documentclass[]{article}
\usepackage{lmodern}
\usepackage{amssymb,amsmath}
\usepackage{ifxetex,ifluatex}
\usepackage{fixltx2e} % provides \textsubscript
\ifnum 0\ifxetex 1\fi\ifluatex 1\fi=0 % if pdftex
  \usepackage[T1]{fontenc}
  \usepackage[utf8]{inputenc}
\else % if luatex or xelatex
  \ifxetex
    \usepackage{mathspec}
  \else
    \usepackage{fontspec}
  \fi
  \defaultfontfeatures{Ligatures=TeX,Scale=MatchLowercase}
\fi
% use upquote if available, for straight quotes in verbatim environments
\IfFileExists{upquote.sty}{\usepackage{upquote}}{}
% use microtype if available
\IfFileExists{microtype.sty}{%
\usepackage{microtype}
\UseMicrotypeSet[protrusion]{basicmath} % disable protrusion for tt fonts
}{}
\usepackage[margin=1in]{geometry}
\usepackage{hyperref}
\hypersetup{unicode=true,
            pdftitle={Hate Crimes in the U.S.},
            pdfauthor={Cecilia Cai},
            pdfborder={0 0 0},
            breaklinks=true}
\urlstyle{same}  % don't use monospace font for urls
\usepackage{graphicx,grffile}
\makeatletter
\def\maxwidth{\ifdim\Gin@nat@width>\linewidth\linewidth\else\Gin@nat@width\fi}
\def\maxheight{\ifdim\Gin@nat@height>\textheight\textheight\else\Gin@nat@height\fi}
\makeatother
% Scale images if necessary, so that they will not overflow the page
% margins by default, and it is still possible to overwrite the defaults
% using explicit options in \includegraphics[width, height, ...]{}
\setkeys{Gin}{width=\maxwidth,height=\maxheight,keepaspectratio}
\IfFileExists{parskip.sty}{%
\usepackage{parskip}
}{% else
\setlength{\parindent}{0pt}
\setlength{\parskip}{6pt plus 2pt minus 1pt}
}
\setlength{\emergencystretch}{3em}  % prevent overfull lines
\providecommand{\tightlist}{%
  \setlength{\itemsep}{0pt}\setlength{\parskip}{0pt}}
\setcounter{secnumdepth}{0}
% Redefines (sub)paragraphs to behave more like sections
\ifx\paragraph\undefined\else
\let\oldparagraph\paragraph
\renewcommand{\paragraph}[1]{\oldparagraph{#1}\mbox{}}
\fi
\ifx\subparagraph\undefined\else
\let\oldsubparagraph\subparagraph
\renewcommand{\subparagraph}[1]{\oldsubparagraph{#1}\mbox{}}
\fi

%%% Use protect on footnotes to avoid problems with footnotes in titles
\let\rmarkdownfootnote\footnote%
\def\footnote{\protect\rmarkdownfootnote}

%%% Change title format to be more compact
\usepackage{titling}

% Create subtitle command for use in maketitle
\providecommand{\subtitle}[1]{
  \posttitle{
    \begin{center}\large#1\end{center}
    }
}

\setlength{\droptitle}{-2em}

  \title{Hate Crimes in the U.S.}
    \pretitle{\vspace{\droptitle}\centering\huge}
  \posttitle{\par}
    \author{Cecilia Cai}
    \preauthor{\centering\large\emph}
  \postauthor{\par}
      \predate{\centering\large\emph}
  \postdate{\par}
    \date{8/10/2019}


\begin{document}
\maketitle

\hypertarget{research-background}{%
\subsection{Research Background}\label{research-background}}

A hate crime, as defined by the FBI, is a crime ``motivated in whole or
in part by an offender's bias against a race, religion, disability,
sexual orientation, ethnicity, gender, or gender identity''. As stated
by the FBI, ``on average, U.S. residents experienced an estimated
250,000 hate crime victimizations each year between 2004 and 2015''.
Previously, the FiveThirtyEight had done a research in 2017, analysing
the reasons that might lead to an increase in hate crimes using U.S.
state-level data. The conclusion they derived at was that ``higher rates
of hate crimes are tied to income inequality'', which remains to be true
before and after the president election.

In our capstone project, we looked into the data related to the
occurrance of hate crimes across the U.S., trying to replicate the
previous FiveThirtyEight analysis using updated hate crime data and
additional regressors, and explore additional potential influencing
factors.

I use average annual hate crimes per 100k population as our dependent
variable, getting the data from the FBI official website, and
specifically, the 2017 report by states. According to the official FBI
website on hate crimes
(\url{https://www.fbi.gov/services/cjis/ucr/hate-crime}), these data was
collected by the Uniform Crime Reporting (UCR) Program, who cooperated
with many local and state law enforcement agencies to track and verify
the data. The incidents are reported by individuals and agencies to the
the UCR voluntarily, who then categorizes and gather these data by the
biases and motivations of the cases. This process potentially bears many
biases. For one thing, the inconsistent in public's perception of hate
crime might result in the disparity in quality of the reported number
across different stats. Although the FBI defination for hate crimes is
quite detailed, the statement is rather conceptual, which is hard to
evaluate by some universal standards. The significant differences in
relevant laws and regulations, as well as their enforcement, across
states, worsen the situation, for they potentially differ the public
understanding and awareness of hate crimes, which affect their tendency
to report a case. On the other hand, state governments may also modify
their reported data for certain political purposes. For another, the
unsystematic and weakly operational methods of collecting data are
likely to cause mistakes or missing values. And indeed, the FBI data set
does not have record from some states such as Hawaii.

\hypertarget{regression}{%
\subsection{Regression}\label{regression}}

In spite of many inevitable biases, it is still possible to get a basic
understanding of the situation of hate crimes in the U.S. by relating
the hate crime data to some other variables from other data sets, and
analyse the correlations and causalities between these variables.

In fact, I established a multivariable linear regression to investigate
the factors related to hate crimes in the U.S.. As instructed, I
referred to the 2016 ACS 5-year data, which is the latest version of the
5-year aggregated data. I also drew some 2017 data from other databases,
for the year 2017 witnessed a great increase in hate crime occurrance,
and would have more representitive data. I ran an initial regression
using the required variables, without adding additional ones.

\% Table created by stargazer v.5.2.2 by Marek Hlavac, Harvard
University. E-mail: hlavac at fas.harvard.edu \% Date and time: Thu, Aug
15, 2019 - 14:08:04 \% Requires LaTeX packages: dcolumn

\begin{table}[!htbp] \centering 
  \caption{Regression Results} 
  \label{} 
\begin{tabular}{@{\extracolsep{5pt}}lD{.}{.}{-3} } 
\\[-1.8ex]\hline 
\hline \\[-1.8ex] 
 & \multicolumn{1}{c}{\textit{Dependent variable:}} \\ 
\cline{2-2} 
\\[-1.8ex] & \multicolumn{1}{c}{avr\_hc} \\ 
\hline \\[-1.8ex] 
 medium\_hs\_income & 0.0004^{***} \\ 
  & (0.0001) \\ 
  & \\ 
 pct\_unemployed & -0.302 \\ 
  & (0.518) \\ 
  & \\ 
 pct\_whi\_pvt & 1.031^{**} \\ 
  & (0.461) \\ 
  & \\ 
 GINIidx & 115.684^{***} \\ 
  & (30.105) \\ 
  & \\ 
 pct\_nonwhite & 0.142 \\ 
  & (0.103) \\ 
  & \\ 
 pct\_hs\_degree & -0.340^{**} \\ 
  & (0.161) \\ 
  & \\ 
 pct\_nonctz & -0.896^{***} \\ 
  & (0.268) \\ 
  & \\ 
 Constant & -69.149^{***} \\ 
  & (17.579) \\ 
  & \\ 
\hline \\[-1.8ex] 
Observations & \multicolumn{1}{c}{50} \\ 
R$^{2}$ & \multicolumn{1}{c}{0.498} \\ 
Adjusted R$^{2}$ & \multicolumn{1}{c}{0.414} \\ 
Residual Std. Error & \multicolumn{1}{c}{3.112 (df = 42)} \\ 
F Statistic & \multicolumn{1}{c}{5.941$^{***}$ (df = 7; 42)} \\ 
\hline 
\hline \\[-1.8ex] 
\textit{Note:}  & \multicolumn{1}{r}{$^{*}$p$<$0.1; $^{**}$p$<$0.05; $^{***}$p$<$0.01} \\ 
\end{tabular} 
\end{table}

As expected, and in line with the conclusion of the previous study
conducted by the FiveThirtyEight, the variables of medium household
income, GINI index, and percentage of white people below poverty line,
all of which as indicators for the economic inequality level of the
state, are positively correlated to average hate crimes. This
strengthens that, with newer data, inequality, especially in terms of
income, is still an important determinant for the frequency of hate
crimes. Moreover, I noticed that the percentage of non citizen in the
state negatively influences the happening of hate crimes, and is quite
significant statistically, which was against my initial intuition. This
result became clearer to me after some additional research on the
contexts of hate crimes, as I learned that most of the hate crimes in
the U.S. are actually conducted by white citizens. In fact, a higher
percentage of non citizens implies a more diversed culture, which might
also explain the negative coefficient. This suggests that inequality in
social status and the degree of inclusiveness of social enviroment are
also related to the rate of hate crimes.

For the two customized independent variables, I used the data from the
statista, respectively per capita alcohol consumption and per capita
state and local government debt. I ran a simple multivariable linear
regression with these veriables as independent variables, without
including any weights, fixed effects, or interaction terms, and the
result is shown in the section below.

\% Table created by stargazer v.5.2.2 by Marek Hlavac, Harvard
University. E-mail: hlavac at fas.harvard.edu \% Date and time: Thu, Aug
15, 2019 - 14:08:04 \% Requires LaTeX packages: dcolumn

\begin{table}[!htbp] \centering 
  \caption{Regression Results} 
  \label{} 
\begin{tabular}{@{\extracolsep{5pt}}lD{.}{.}{-3} } 
\\[-1.8ex]\hline 
\hline \\[-1.8ex] 
 & \multicolumn{1}{c}{\textit{Dependent variable:}} \\ 
\cline{2-2} 
\\[-1.8ex] & \multicolumn{1}{c}{avr\_hc} \\ 
\hline \\[-1.8ex] 
 medium\_hs\_income & 0.0003^{**} \\ 
  & (0.0001) \\ 
  & \\ 
 pct\_unemployed & -0.527 \\ 
  & (0.480) \\ 
  & \\ 
 pct\_whi\_pvt & 1.280^{***} \\ 
  & (0.433) \\ 
  & \\ 
 GINIidx & 64.057^{*} \\ 
  & (34.019) \\ 
  & \\ 
 pct\_nonwhite & 0.209^{**} \\ 
  & (0.098) \\ 
  & \\ 
 pct\_hs\_degree & -0.300^{*} \\ 
  & (0.149) \\ 
  & \\ 
 pct\_nonctz & -0.887^{***} \\ 
  & (0.247) \\ 
  & \\ 
 alcohol\_pc & 1.411^{*} \\ 
  & (0.791) \\ 
  & \\ 
 debts & 0.001^{**} \\ 
  & (0.0002) \\ 
  & \\ 
 Constant & -49.932^{**} \\ 
  & (19.265) \\ 
  & \\ 
\hline \\[-1.8ex] 
Observations & \multicolumn{1}{c}{50} \\ 
R$^{2}$ & \multicolumn{1}{c}{0.600} \\ 
Adjusted R$^{2}$ & \multicolumn{1}{c}{0.510} \\ 
Residual Std. Error & \multicolumn{1}{c}{2.847 (df = 40)} \\ 
F Statistic & \multicolumn{1}{c}{6.656$^{***}$ (df = 9; 40)} \\ 
\hline 
\hline \\[-1.8ex] 
\textit{Note:}  & \multicolumn{1}{r}{$^{*}$p$<$0.1; $^{**}$p$<$0.05; $^{***}$p$<$0.01} \\ 
\end{tabular} 
\end{table}

In this regression, both per capita alcohol consumption and per capita
state and local government debt show positive influence to the average
hate crimes per 100k population, with significant levels at respectively
90\% and 95\%. Besides, all other independent variables, except the
percentage of unemployment rate, are showing high significant level,
especially the percentage of noncitizen rate. The inclusion of these two
variables also slightly increases the R-squared value, suggesting that
they do improve the regression model without bringing in much noises.

Perceiving from common cognition perspective, the variable of per capita
state and local government debt evaluates the economic status of the
state, which affects the operation of the goverment and the general
living quality of its people. The higher the debts, which indicates that
the state might be lacking money for its administrations and
investments, the more likely hate crimes would occur. This suggests that
the rate of hate crimes correlates with the capability and effectiveness
of government administration.

As for alcohol consumption, which is considered to be likely to increase
one's irrationality and tendency to cause chaos, it holds the potential
for increasing hate crimes. Therefore, lower stability of the society as
a whole may also explain for higher occurrance of hate crimes.

Curious about other influencing factors, I did an additional regression,
taking out the two additional independent variables and filling in with
the percentage of very religious people in each state. I chose this
variable because many studies on hate crimes report a strong relation
between hate crimes and religions. Religion might itself be a cause for
hate crimes, and the variaty of religions to some degree reflects the
diversity of the society.

Regression Results

Dependent variable:

avr\_hc

medium\_hs\_income

0.0002**

(0.0001)

pct\_unemployed

0.157

(0.312)

pct\_whi\_pvt

0.465*

(0.258)

GINIidx

33.949*

(18.313)

pct\_nonwhite

-0.032

(0.066)

pct\_hs\_degree

-0.087

(0.092)

pct\_nonctz

-0.191

(0.168)

Very\_religious

-0.006

(0.042)

Constant

-23.695**

(10.951)

Observations

49

R2

0.319

Adjusted R2

0.183

Residual Std. Error

1.694 (df = 40)

F Statistic

2.341** (df = 8; 40)

Note:

\emph{p\textless{}0.1; \textbf{p\textless{}0.05; }}p\textless{}0.01

Suprisingly, the inclusion of this variable reduces the significance of
many other independent variables, and itself is not showing statistical
significance either. It also largely pulls down the Adjusted R-squred,
causing an overfit model. This implies that the independent variable
Very\_religious might not have much explanatory effect, or that it is
related to too many other independent variables, which results in a
model that is too complex to be described by the existing coefficients.
However, despite the overfitting consequence, the variables medium
household income, pecentage of white people under the poverty line, and
the GINI index still maintain a moderate significance to the regression,
furthur justify that income inequality is indeed essencial in causing
hate crimes.

\hypertarget{plots}{%
\subsection{Plots}\label{plots}}

\hypertarget{section}{%
\paragraph{1}\label{section}}

The first plot I made is a bubble plot showing the correlation between
GINI index, percentage of Non-citizen and average hate crimes per 100k.
The x and y axises respectively represent the percentage of non-citizen
and the GINI index, while the size and color of the bubbles jointly
suggest the average occurrences of hate crimes per 100k.

\begin{center}\includegraphics{Capstone_files/figure-latex/customized_plot-1} \end{center}

From this plot, we can see that states with GINI index around 0.45-0.48
tend to witness more hate crimes, while the percentage of non citizen is
not showing a clear corelation to average hate crimes holding the GINI
index fixed. It also suggests that the GINI index of states with higher
percentage of non citizen are originally higher.

\hypertarget{section-1}{%
\paragraph{2}\label{section-1}}

The second plot is a map showing the distribution of average hate crimes
by states across the United States. Noted that the data from Hawaii is
missing for this map plot.

\begin{center}\includegraphics{Capstone_files/figure-latex/map_plot-1} \end{center}

\hypertarget{section-2}{%
\paragraph{3}\label{section-2}}

The last plot is Line chart displaying the change in aggregate number of
hate crimes per 100k over time, from 2008 to 2017.

\begin{center}\includegraphics{Capstone_files/figure-latex/non_map_plot-1} \end{center}

\hypertarget{conclusions-and-reflections}{%
\subsection{Conclusions and
Reflections}\label{conclusions-and-reflections}}

The results of my regressions, as well as the plots I made, suggest that
the distribution of wealth, the condition of government administration,
and the composition of residents, are all important determinants for
hate crimes. Extending the conclusion from the previous study by
FiveThirtyEight, not only income inequality, but also inequality in
other aspects such as social status and cultural backgroud, increase
rates of hate crimes.

Although many of the variables I tested, which are not included in this
report, such as the divorced rate by states, the number of prisoners by
states, and etc., fail to show statistically significant results, it is
not necessarily because they have no effects on hate crimes, but rather
that they are too closely related to some other variables such that
adding them would bring in too much omitted variable bias to the
regression. This reminds me to be always cautious about choosing
independent variables, and in particular, be mindful of their potential
inner relationship. In addition, when analysing and reporting the
result, we should combine the information we get from data with our
real-life knowledge, rather than be hashy to draw conclusions directly
from the data. As a matter of fact, data might be misleading for most of
the time, and we would never understand the situation thoroughly without
looking for background knowledge and cultural contexts.


\end{document}
